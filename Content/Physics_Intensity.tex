Irradiance, Intensity is impact at receiver (observer in distance b):

\vspace{2ex}\noindent
$\mathscr{I} \equiv$ Irradiation, Intensity, "straight impact" at receiver (observer surface orthogonal to  ray beam, $\mathscr{A}_b \perp R$) \index{Irradiation} \index{Intensity} \index{Power} \index{Flux} \\
$\Phi \equiv$ Power (Flux) = $\frac{W}{t} = \frac{F \cdot x}{t} = \frac{dE}{dt} = mva = pa$ \\
$\mathscr{S} \equiv$ Poynting-Vector, Radiation \index{Poynting-Vector} \index{Radiation} \\
$\mathscr{A}_b \equiv$ Area surface at observer (receiver) \\
$\mathscr{A}_g \equiv$ Area surface at source (sender) \\
$\Omega_g \equiv$ Solid Angle at source (sender) \index{Solid Angle} \\
$\Omega_b \equiv$ Solid Angle at observer (receiver) \\
$\mathscr{J} \equiv$ Flux Density (Radiant Intensity), "angular emission" at source (sender) \index{Flux Density} \index{Radiant Intensity} \\
$\mathscr{L} \equiv$ Radiance \index{Radiance} \\
$R \equiv$ Distance, ray beam length
\vspace{2ex}

$
\mathscr{I} = 
\mathscr{I}_{\alpha} = 
\langle $w$_{EM} \rangle \cdot c = 
\langle $w$_{EM} \cdot c_{ph} \rangle = 
\frac{1}{2} \frac{c}{\eta} \epsilon \mathscr{E}_0^2 = 
\frac{1}{2} c_{ph} \epsilon_0 \mathscr{E}_0^2 = 
\frac{1}{\mu_0} | \mathscr{E} \times \mathscr{B} | = 
\langle | \mathscr{E} \times \mathscr{B} | \rangle = 
\langle | \mathscr{S} | \rangle =  
\langle \Phi \rangle =  
\frac{d\Phi}{d\mathscr{A}_b} = 
\frac{d\Phi}{d\Omega_g} \frac{\cos(\varepsilon_b)}{R^2} = 
\mathscr{J} \frac{\cos(\varepsilon_b)}{R^2} 
\stackrel{\mathscr{A}_b \perp \mathscr{S}, \varepsilon_b = 0}{=}
\frac{\mathscr{J}}{R^2} 
\stackrel{\Omega_b=small, \mathscr{S}=const.}{=} 
\int\limits_{\mathscr{A}_g} \mathscr{L} \frac{\cos(\varepsilon_g)}{R^2} d\mathscr{A}_g = 
\int\limits_{\Omega_b} \mathscr{L} d\Omega_g = 
4\pi \cdot \mathscr{L}_B 
$ \footnote{$\mathscr{L}_B = \textnormal{Blackbody Radiance}$} 
\index{Balckbody Radiance}
\newline \vspace{2ex}

 $
\mathscr{I}_0
\stackrel{\mu_{r}=1}{=}
c \epsilon_{0} \left( \frac{\mathscr{E(\vartheta)}}{e^{ t (\omega t - kr) }}  \frac{sin(\frac{1}{2} \Delta \varphi )}{sin(\frac{N}{2} \Delta \varphi) }   \right )^2 =
\mathscr{I}_{A_r} + \mathscr{I}_{A_t} = 
\mathscr{I}_{\vartheta} + \mathscr{I}_{\beta} = 
\mathscr{I}_{A_t,max} =
\mathscr{I}_{\beta_t,max} = 
\mathscr{E}_0^2
$
\vspace{2ex}

$
\mathscr{I}_{max}
\stackrel{\mu_{r}=1}{=}
\mathscr{I}_{\alpha=0=\beta} = 
\mathscr{I}(0) = 
N^2 \mathscr{I}_{0}
\stackrel{\mu_{r}=1}{=}
\left( N \mathscr{E}_0 \right)^2
$


\vspace{3ex} \noindent
Transmitted Intensity (Airy-Function) in refraction:
\newline \vspace{2ex}
$
\mathscr{I}_{\beta_t} = \mathscr{I}_{A_t} = 
\frac{1}{1 + \frac{4 a_r^2}{(1-a_r^2)^2} sin^2(\frac{\Delta\varphi}{2}) } = 
\frac{1}{ 1 + \mathcal{F} \cdot sin^2(\frac{\Delta\varphi}{2}) }
$
\newline \vspace{2ex}
$
\mathscr{I}_{A_t}(\Delta\varphi) = 
\mathscr{E}^*\mathscr{E} = 
\mathscr{E}_{0}^2 \frac{ \left( 1 - \left( \frac{\mathscr{E}_{0,\vartheta}}{\mathscr{E}_{0,\alpha}} \right)^2 \right)^2 }{ \left( 1 - \left( \frac{\mathscr{E}_{0,\vartheta}}{\mathscr{E}_{0,\alpha}} \right)^2 \right)^2 } \frac{1}{1+\frac{ 4 \left( \frac{\mathscr{E}_{0,\vartheta}}{\mathscr{E}_{0,\alpha}} \right)^2 }{  \left( 1 - \left( \frac{\mathscr{E}_{0,\vartheta}}{\mathscr{E}_{0,\alpha}} \right)^2 \right)^2 } \sin^2(\frac{\Delta\varphi}{2}) } = 
\mathscr{E}_{0}^2 \frac{ \left( 1 - a_r^2 \right)^2 }{ \left( 1 - a_r^2 \right)^2 } \frac{1}{1+\frac{ 4a_r^2 }{  \left( 1 - a_r^2 \right)^2 } \sin^2(\frac{\Delta\varphi}{2}) } = 
\mathscr{E}_0^2 \mathscr{I}_{A_t} = 
\mathscr{I}_{max} \mathscr{I}_{A_t}
$
\newline \vspace{2ex}
$
\mathscr{I}_{\beta_t,min} =
\mathscr{I}_{A_t,min} =
\mathscr{I}_{A_t,max} \cdot \frac{1}{1 + {\frac{4 a_r^2}{(1-a_r^2)^2}} } = 
\mathscr{I}_{\beta_t,max} \cdot \frac{1}{1 + \mathcal{F}} = 
\frac{ \mathscr{E}_0^2}{1 + \mathcal{F} }
$


\vspace{3ex} \noindent
Absorbed Intensitiy:
\newline \vspace{2ex}
$
\mathscr{I} = 
\mathscr{I}_{\beta_a} = 
\mathscr{I}_0 e^{-al} 
\stackrel{\mu_r = 1}{=}
\frac{\mathscr{E}_0^2}{e^{al}}	
\qquad	a = \textnormal{absorbtion coefficient}	
\qquad	l = \textnormal{direction}
$


\vspace{3ex} \noindent
Reflected Intensity:
\newline \vspace{2ex}
$
\mathscr{I}_{\vartheta}
\stackrel{\mu_{r}=1}{=}
c \epsilon_0 \langle \mathscr{E(\vartheta)}^2 \rangle =
\mathscr{I}_0 \frac{sin^2 \left (\frac{N}{2} \Delta \varphi \right)}{sin^2 \left( \frac{1}{2} \Delta \varphi \right)} = 
\mathscr{I}_0 \frac{sin^2 \left( N \pi \frac{d}{\lambda} sin(\vartheta) \right) }{sin^2 \left( \pi \frac{d}{\lambda} sin(\vartheta) \right) } 
$


\vspace{3ex} \noindent
Difracted Intensity:
\newline \vspace{2ex}
$
\mathscr{I}_{\theta}
\stackrel{\mu_{r}=1}{=}
c \epsilon_0 \langle \mathscr{E}^2 \rangle =
\underbrace{\mathscr{I}_0}_{Max.}
\underbrace{ \left( \frac{sin(\beta)}{\beta} \right)^2 }_{Modulation}
\underbrace{ \left( \frac{sin(N \alpha)}{sin(\alpha)} \right)^2 }_{Variation}
$
\vspace{2ex}
$
\newline 
\beta = k \frac{b}{2}sin(\theta)	\qquad	\alpha = k \frac{a}{2}sin(\theta) 
\newline
a = \textnormal{particle distance}	\qquad	b = \textnormal{particle size}
$


\vspace{5ex} \noindent
Hypothesis: 
$ \left( \frac{\mathscr{J}}{R}  \right)^2 = \mathscr{J} \cdot \mathscr{I} $
$
\Rightarrow \text{ray length (reach) }
R = \sqrt{ \frac{\mathscr{J}}{\mathscr{I}} }
$	\index{ray length}
