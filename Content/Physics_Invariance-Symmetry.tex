
If for an optical process \cite{pfeiler_experimentalphysik_Ed1_V4_58} \\
1. the absorbtion is neglectable and \\
2. there is no change of the polarisation state of the radiation, \\
then the \textbf{relationship} between \\
a) the oscillation of the electric current at a certein point in space and its \\
b) induced electric field at a differnt point in space \\
stays \textbf{unchanged} if those two points in space are switched (those two points where the oscillation of the electric current and the measured electric field occure).

Thus there is a symmetry in the wave propagation of electromagnetic radiation, when the two locations between \\
a) the source of radiation and \\
b) the observed detection \\
are exchanged.

This is the \textbf{reversibility} of a radiation path, which in physics is called the "Reciprocity Principle".

According to the physics of Wave Optics (Refraction Index, Dispersion and Absorbion, Lightspeed), in a medium (matter) with refraction index $\eta^2 = \epsilon_r\mu_r 
\stackrel{dispersion}{=} 
\eta^2(\lambda) = \eta^2(\omega)$ the \textbf{frequency } $\nu = \frac{\omega}{2\pi} = \frac{\eta k v}{2\pi} = \frac{\eta v}{\lambda}
\stackrel{vacuum}{=}
 \frac{kc}{2\pi} = \frac{k}{2\pi}\frac{1}{\sqrt{\epsilon_0\mu_0}} = \frac{c}{\lambda} $ of the electromagnetic wave always remains unchanged if no external force is applied - it is \textbf{invariant}; while its wavelength $\lambda = \frac{2\pi}{k} $ might shorten getting slower (where k = wavenumber, and $v_{max}=c$ = lightspeed).

\vspace{5ex}
\begin{hypothesis}[Reversibility of Information]
	Therefore, energy $E(\nu)$ is conserved - meaning: Information (energy) transmitted with the propagation of electromagnetic waves at a specific frequency remains stable because the optical process is reversible and thus has no change in entropy.
	\end{hypothesis}
